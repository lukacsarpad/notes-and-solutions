\documentclass[a4paper,12pt]{article}
\usepackage[english]{babel}
\usepackage[utf8]{inputenc}
\usepackage{t1enc}
%\usepackage[T1]{fontenc}
\usepackage{floatflt}
\usepackage{graphicx}
\usepackage{psfrag}
\usepackage{bbm}
\usepackage{amsmath}
\usepackage{amssymb}
\usepackage{slashed}
%\usepackage{showkeys}
\usepackage{hyperref}
\usepackage{ifthen}
\usepackage{subcaption}
\usepackage{epstopdf}



\hoffset=-5.0mm
\voffset=-1.9mm
%
\evensidemargin=0cm
\oddsidemargin=0cm
\topmargin=0cm%
\headheight=0cm%
\headsep=0cm%
\marginparsep=0cm%
\marginparwidth=0cm%
\textheight=24cm
\textwidth=17cm
\special{papersize=210mm,297mm}%

\def\d{\mathrm{d}}
\def\e{\mathrm{e}}
\def\imagi{\mathrm{i}}
\def\ellop{\mathop{{\sf L}}}
\def\forrasfile#1{ (see {\tt #1})}
\def\lag{{\mathcal{L}}}
\def\lap{\mathop{\Delta}}
\def\kihagy#1{}
\def\sn{\mathop{\text{sn}}}
\def\cn{\mathop{\text{cn}}}
\def\dn{\mathop{\text{dn}}}
\def\zb{\ensuremath{\bar{z}}}
\def\sign{\mathop{\text{sign}}}
\def\xii#1{\xi^{(#1)}}
\def\xij#1#2{{\xi^{(#1)}_{#2}}}
\def\op#1{{\sf #1}}
\def\tphi{\ensuremath{\tilde{\phi}}}
\def\tphi{\ensuremath{\tilde{\phi}}}
\renewcommand\Re{\mathop{\text{Re}}}
\renewcommand\Im{\mathop{\text{Im}}}
\def\Tr{\ensuremath{\mathop{\rm Tr}}}
\def\pa{\partial}


\newcommand{\doi}[1]{\href{http://dx.doi.org/#1}{DOI: #1}}%
\newcommand{\doix}[2]{\href{http://dx.doi.org/#2}{#1}}%
\newcommand{\arxiv}[2][]{%
  \ifthenelse{\equal{#1}{}}{%
    \href{http://arxiv.org/abs/#2}{\texttt{arXiv:#2}}%
  }{%
    \href{http://arxiv.org/abs/#2}{\texttt{arXiv:#2 [#1]}}%
  }%
}%


%opening
\title{Some notes on classical fields}
\author{Árpád Lukács}

\begin{document}

\maketitle

\section{Introduction}\label{sec:intro}
In this note we shall be concerned with the basics of classical field theory. The starting point will be to extend Par.\ 32 of Ref.\ \cite{LL2} to include the spin tensor, conservation of angular-momentum and the Belinfante-Rosenfeld procedure, following Ref.\ \cite{wentzel}. For Dirac matrix notations, we follow Ref.\ \cite{BD, itzykson}

We shall assume that the fields $\phi_a$ (this may include components of vectors, covectors, tensors, spinors, etc.) satisfy field equations stemming from the principle of least action, with an action as the integral of the Lagrangian,
\begin{equation}\label{eq:action}
 S = \int \d^d x \lag\,, \lag=\lag(\partial_\mu \phi_a, \phi_a)\,,
\end{equation}
and in what follows, exploit the symmetries of the action.

The field equations are derived using the usual procedure, of demanding that the functional derivative of the action $S$ vanish. The latter is obtained by adding a small variation to the fields, $\delta\phi_a$, and calculating the linear contribution to $S$, $\delta S$, performing partial integration, and dropping boundary terms, yielding
\begin{equation}\label{eq:fieldeq}
 \partial_\mu \overline{\frac{\partial \lag}{\partial \partial_\mu \phi_a}} = \overline{\frac{\partial\lag}{\partial \phi_a}}\,.
\end{equation}
Eq.\ (\ref{eq:fieldeq}) holds for all $a$.

Note about notation: we shall use the conventions of Ref.\ \cite{LL2}, with the role of Greek and Latin indices reversed. The metric is $\mathop{\rm diag}(+,-,-,-)$ and Greek indices run from 0 to $d-1$. The index $a$ in Eq.\ (\ref{eq:fieldeq}) runs over all field components, but we shall also use latin indices running over $1,\dots, d-1$ (spatial dimensions).

Another useful notation that is usually omitted in Eq.\ (\ref{eq:fieldeq}) is the overline. Here, it denotes taking the composition of the function with a solution of the field equations, e.g.,
\begin{equation}
 \overline\lag = \lag(\partial_\mu \phi_a, \phi_a)\,,
\end{equation}
which, in a modern mathematical notation would be $\overline{\lag} = \lag \circ(D\phi,\phi)$, and for Eq.\ (\ref{eq:fieldeq}),
\begin{equation}\label{eq:fieldeqBourbaki}
 D [ D_1 \lag \circ (D\phi, \phi)] = D_2 \lag \circ (D\phi, \phi)\,.
\end{equation}



\section{Internal symmetries}\label{sec:int}
Let us first demonstrate how internal symmetries yield conservation laws. Internal symmetries are symmetries that transform the field $\phi_a$ in such a way, that the new field values only depend on the old ones at the same point. 

We shall demonstrate here, how to each one-parameter subgroup of the internal symmetries of the theory corresponds a conserved current. An element of a one parameter subgroup, corresponding to a small parameter $\epsilon$ transforms the fields as
\begin{equation}\label{eq:IntSP}
 \delta \phi_a = \epsilon M_a{}^b \phi_b\,,
\end{equation}
and what we shall do is assume that the Lagrangian is invariant under the transformation. Let us now take the derivative,
\begin{equation}\label{eq:DLIS}
 0=\partial_\epsilon \overline{\lag}=M_a{}^b\left[\overline{\frac{\partial{\lag}}{\partial\partial_\lambda\phi_a} }\partial_\lambda\phi_b + \overline{\frac{\partial\lag}{\partial\phi_a}}\phi_b\right] = M_a{}^b \partial_\lambda\left[\overline{\frac{\partial{\lag}}{\partial\partial_\lambda\phi_a}}\phi_b\right]=\partial_\lambda J^\lambda\,,
\end{equation}
where we have used the field equations (\ref{eq:fieldeq}) in the last step;
\begin{equation}\label{eq:JIS}
 J^\mu = M_a{}^b\overline{\frac{\partial{\lag}}{\partial\partial_\lambda\phi_a}}\phi_b\,.
\end{equation}

Conservation of the current: consider the integral
\begin{equation}\label{eq:Jcons}
 Q = \int \d^3 x Q^0\,, \dot{Q} = \int \d^2 x \dot{Q^0} = -\int\d^3 x \partial_i J^i = 0\,,
\end{equation}
which is seen by using Gauss' theorem to turn the last integral into a surface integral over a large sphere, and assuming that the fields vanish far away.


\section{Canonical energy momentum tensor}\label{sec:EMcan}
The canonical energy-momentum tensor is derived from the fact that the Lagrangian does not depend \emph{explicitly} on the position. This means, that of course, $\overline{\lag}$ does depend on time, but $\lag$ itself not, i.e., it is a two-variable function (one variable is to be replaced by $\partial_\mu\phi_a$ and one by $\phi_a$), i.e., $\overline{\lag}=\lag(\partial_\mu\phi_a, \phi_a)$ and not $\lag(\partial_\mu\phi_a, \phi_a, x)$. Therefore, we may calculate the derivative of $\overline{\lag}$ as follows,
\begin{equation}\label{eq:DST}
 \partial_\mu \overline{\lag} = \overline{\frac{\partial\lag}{\partial\partial_\lambda\phi_a}} \partial_\mu\partial_\lambda\phi_a + \overline{\frac{\partial\lag}{\partial\phi_a}}\partial_\mu \phi_a = 
 \partial_\lambda\left[ \overline{\frac{\partial\lag}{\partial\partial_\lambda\phi_a}} \partial_\mu\phi_a\right]\,,
\end{equation}
or, brought on one side,
\begin{equation}\label{eq:DST0}
 0 =  \partial_\lambda\left[ \overline{\frac{\partial\lag}{\partial\partial_\lambda\phi_a}} \partial_\mu\phi_a -\delta^\lambda_\mu \overline{\lag}\right] =\partial_\lambda\Theta^\lambda{}_\mu\,,
\end{equation}
where the latter is the canonical stress-energy (or energy-momentum) tensor,
\begin{equation}\label{eq:EMcan}
 \Theta^\mu{}_\nu = \overline{\frac{\partial\lag}{\partial\partial_\mu\phi_a}} \partial_\nu\phi_a -\delta^\mu_\nu \overline{\lag}\,.
\end{equation}

The interpretation of $\Theta^\mu{}_\nu$ is that there are a set of conserved quantities,
\begin{equation}\label{eq:momcons}
 P_\mu = \int \d^3 x \Theta^0{}_\mu\,,
\end{equation}
which are interpreted as energy and momentum. The conservation is shown the same way as for the charge, see eq.\ (\ref{eq:Jcons}) in sec.\ \ref{sec:int}.

\section{Angular momentum and spin}\label{eq:angmom}
We shall consider angular momentum, or more generally, the conserved quantities corresponding to Lorentz invariance of the action. Angular momentum is the part of the result corresponding to rotations.

To proceed we must first consider the change of fields upon applying a Lorentz transformation. First, scalars, vectors, tensors, {\sl etc.}.

What is a Lorentz transformation? A transformation of Minkowski space that preserves the scalar product, i.e.,
\begin{equation}\label{eq:Lorentz}
 x^\mu{}'=\Lambda^\mu_\nu x^\nu\,,\quad x_1^\mu{}' x_{2,\mu}' = x_1^\mu x_{2,\mu}\,,
\end{equation}
or, introducing the metric, $g={\rm diag}(+,-,-,-)$, a transformation such that
\begin{equation}\label{eq:Lorentz2}
 g_{\mu\nu}\Lambda^\mu{}_\sigma \Lambda^\nu{}_\rho = g_{\sigma\rho}\,.
\end{equation}
Consider now an ``infinitesimal'' Lorentz transformation, i.e., one for which
\begin{equation}\label{eq:InfiLorentz}
 \Lambda^\mu_\nu = \delta^\mu_\nu + \omega^\mu_\nu + \dots\,,
\end{equation}
and expand the condition (\ref{eq:Lorentz2}), yielding
\begin{equation}\label{eq:InfiLorentz2}
 \omega_{\mu\nu}+\omega_{\nu\mu} = 0\,,
\end{equation}
i.e., ``infinitesimal'' Lorentz transformations are obtained by raising one index of an antisymmetric matrix.

On some indexed quantities, Lorentz transformations are represented, i.e., they may act linearly,
\begin{equation}\label{eq:rep}
 v_a' = D_a{}^b(\Lambda)v_b\,,
\end{equation}
and, for infinitesimal transformations, this may again be expanded,
\begin{equation}
 \delta v_a = \omega_{\mu\nu}D^{\mu\nu}{}_a{}^b v_b\,,
\end{equation}
where the $D$ may be chosen such that it is antisymmetric in the indices $\mu$ and $\nu$.

Of course, for scalars $D^{\mu\nu} = 0$.

Let us determine this matrix for a vector! We already know how a vector transforms, just need to factor $\omega_{\mu\nu}$:
\begin{equation}\label{eq:vectrf}
 \delta v^\alpha = \omega^\alpha{}_\beta v^\beta = g^{\alpha\mu}\omega_{\mu\nu}\delta^\nu_\beta v^\beta = \omega_{\mu\nu}\frac{1}{2}\left( g^{\mu\alpha}\delta^\nu_\beta - g^{\nu\alpha}\delta^\mu_\beta\right) v^\beta\,,
\end{equation}
and we may conclude that
\begin{equation}\label{eq:DefRep}
 D^{\mu\nu\alpha}{}_\beta = \frac{1}{2}\left( g^{\mu\alpha}\delta^\nu_\beta - g^{\nu\alpha}\delta^\mu_\beta\right)\,,
\end{equation}
which is also termed the \emph{defining representation} of the Lorentz group.

For Dirac spinors,
\begin{equation}\label{eq:DiracRep}
 D^{\mu\nu}_{\rm D}\psi = -\frac{\imagi}{4} \sigma^{\mu\nu}\psi\,,
\end{equation}
where $\sigma^{\mu\nu}=(\imagi/2)[\gamma^\mu,\gamma^\nu]=(\imagi/2)(\gamma^\mu\gamma^\nu-\gamma^\nu\gamma^\mu)$, and $[,]$ denotes the commutator, $\{,\}$ the anticommutator, and $\gamma^{\mu}$ the Dirac matrices, i.e., four $4\times 4$ matrices that satisfy $\{\gamma^\mu,\gamma^\nu\}=2g^{\mu\nu}$.

For the adjoint Dirac spinor, $\bar{\psi}=\psi^\dagger \gamma^0$, using the usual normalisation of the Dirac matrices, $\gamma^{\mu\dagger}=\gamma^0 \gamma^\mu \gamma^0$, we obtain
\begin{equation}\label{eq:AdjDiracRep}
 D^{\mu\nu}_{\rm aD} \bar{\psi} = \frac{\imagi}{4}\bar{\psi} \sigma^{\mu\nu}\,.
\end{equation}


The transformation of \emph{fields} is calculated from their tensorial character and space-time dependece,
\begin{equation}\label{eq:fieldtrf}
 \phi_a(x') ' = D_a{}^b(\Lambda) \phi_b(x)\,,
\end{equation}
which may be linearised to yield
\begin{equation}\label{eq:InfFieldtrf}
 \delta \phi_a = \omega_{\mu\nu}\left[ D^{\mu\nu}{}_a{}^b \phi_b - D^{\mu\nu\alpha}{}_{\beta}x^\beta \partial_\alpha \phi_a \right]\,.
\end{equation}

Let us now turn to deriving the quantities corresponding to Lorentz invariance of the action (\ref{eq:action}). Lorentz invariance is manifested by the fact, that the Lagrangian is a scalar,
\begin{equation}\label{eq:LorentzDeltaL}
 \delta\lag = -\omega_{\mu\nu}D^{\mu\nu\alpha}{}_\beta x^\beta \partial_\alpha\overline{\lag}\,.
\end{equation}
On the other hand, we may use the chain rule, yielding
\begin{equation}\label{eq:LorentzDeltaLChain}
\begin{aligned}
 \delta\lag &= \overline{\frac{\partial\lag}{\partial\partial_\lambda\phi_a}}\delta\partial_\lambda\phi_a + \overline{\frac{\partial\lag}{\partial\phi_a}}\delta\phi_a\\
 &= 
 \omega_{\mu\nu}\left[\overline{\frac{\partial\lag}{\partial\partial_\lambda\phi_a}} \left( D^{\mu\nu}{}_a{}^b \partial_\lambda\phi_b -D^{\mu\nu\alpha}{}_\beta x^\beta \partial_\alpha \partial_\lambda\phi_a + D^{\mu\nu}{}_\lambda{}^\beta \partial_\beta \phi_a\right)
 \right.\\
 &\quad\quad\left.
 +\overline{\frac{\partial\lag}{\partial\phi_a}}\left(D^{\mu\nu}{}_a{}^b \phi_b-D^{\mu\nu\alpha}{}_\beta x^\beta\partial_\alpha \phi_a\right)
 \right] \\
 &= \omega_{\mu\nu}\left[-D^{\mu\nu\alpha}{}_\beta x^\beta \partial_\lambda \left( \overline{\frac{\partial\lag}{\partial\partial_\lambda\phi_a}}\partial_\alpha\phi_a\right)+D^{\mu\nu}{}_a{}^b \partial_\lambda \left( \overline{\frac{\partial\lag}{\partial\partial_\lambda\phi_a}}\phi_b\right)\right]\,.
\end{aligned}
\end{equation}
Let us now take the difference of eqs.\ (\ref{eq:LorentzDeltaL}) and (\ref{eq:LorentzDeltaLChain}), yielding
\begin{equation}\label{eq:LorentzDeltaL0}
\begin{aligned}
 0 &= \omega_{\mu\nu}\left[-D^{\mu\nu\alpha}{}_\beta x^\beta \partial_\lambda \left( \overline{\frac{\partial\lag}{\partial\partial_\lambda\phi_a}}\partial_\alpha\phi_a-\delta^\lambda_\alpha \overline{\lag}\right)
 %
 +D^{\mu\nu}{}_{\alpha}{}^\beta \overline{\frac{\partial\lag}{\partial\partial_\alpha\phi_a}}\partial_\beta \phi_a
 %
 +D^{\mu\nu}{}_a{}^b \partial_\lambda \left( \overline{\frac{\partial\lag}{\partial\partial_\lambda\phi_a}}\phi_b\right)\right]\,,\\
 &=\omega_{\mu\nu}\left[
 \frac{1}{2} D^{\mu\nu}{}_{\alpha\beta}\left( x^\alpha \partial_\lambda \Theta^{\lambda\beta}-x^\beta\Theta^{\lambda\alpha}+\Theta^{\alpha\beta}-\Theta^{\beta_\alpha}\right)
 +D^{\mu\nu}{}_a{}^b \partial_\lambda \left( \overline{\frac{\partial\lag}{\partial\partial_\lambda\phi_a}}\phi_b\right)
 \right]\\
 &= \omega_{\mu\nu}\frac{1}{2}\partial_\lambda \left(M^{\lambda\mu\nu}_{\rm o}+S^{\lambda\mu\nu}\right)\,,
 \end{aligned}
\end{equation}
where we noticed, that the first expression in round brackets is the canonical stress-energy tensor $\Theta^\lambda{}_\alpha$! Next, we have raised and lowered indices to factorise the $D^{\mu\nu}{}_{\alpha\beta}$, and antisymmetrised. The resulting tensors $M_{\rm o}$ and $S$ are the \emph{orbital angular momentum} and \emph{spin} tensors, respectively. Here
\begin{equation}\label{eq:OrbAM}
 M_{\rm o}^{\lambda\mu\nu} = D^{\mu\nu}{}_{\alpha\beta}\left(x^\alpha\Theta^{\lambda\beta}-x^\beta\Theta^{\lambda\alpha}\right) = x^\mu \Theta^{\lambda\nu} - x^\nu\Theta^{\lambda\mu}\,,
\end{equation}
which is interpreted as orbital angular momentum density, as from eq.\ (\ref{eq:EMcan}), $\Theta^{0\mu}$ is interpreted as momentum density, and therefore $x^\mu\Theta^{0\nu}- x^\nu\Theta^{0\mu}$ is orbital angular momentum. The other term in the angular momentum is spin,
\begin{equation}\label{eq:SpinAM}
 S^{\lambda\mu\nu} = 2D^{\mu\nu}{}_a{}^b \overline{\frac{\partial\lag}{\partial\partial_\lambda\phi_a}}\phi_b\,.
\end{equation}
For future reference, we note that $S^{\lambda\mu\nu}=-S^{\lambda\nu\mu}$,
and
\begin{equation}\label{eq:DivSAM}
 \partial_\lambda S^{\lambda\mu\nu} =-\partial_\lambda M_{\rm o}^{\lambda\mu\nu}= 
 -\left( \Theta^{\mu\nu} - \Theta^{\nu\mu}\right)\,.
\end{equation}
We also introduce the notion of the total angular momentum density,
\begin{equation}\label{eq:AM}
 M^{\lambda\mu\nu} = M_{\rm o}^{\lambda\mu\nu}+S^{\lambda\mu\nu}\,,
\end{equation}
which is conserved, from eq.\ (\ref{eq:LorentzDeltaL0}),
\begin{equation}\label{eq:AMcons}
\partial_\lambda M^{\lambda\mu\nu} = 0\,.
\end{equation}



\section{Symmetrised stress-energy tensor}
We now intend to construct the Belinfante-Rosenfeld symmetric stress-energy tensor, following Ref.\ \cite{wentzel}. This tensor should satisfy the following criteria: (i) its integral should be the momentum
\begin{equation}\label{eq:BRc1}
 \int T^{0\mu}\d^3 x = P^\mu = \int \Theta^{0\mu} \d^3 x\,,
\end{equation}
(ii) it should be symmetric,
\begin{equation}\label{eq:BRc2}
 T^{\mu\nu} = T^{\nu\mu}\,
\end{equation}
and (iii) it should be conserved,
\begin{equation}\label{eq:BRc3}
 \partial_\mu T^{\mu\nu} = 0\,.
\end{equation}

We can fulfill the first condition, eq.\ (\ref{eq:BRc1}) if the Belinfante-Rosenfeld tensor differs from the canonical stress-energy tensor by a four-divergence,
\begin{equation}\label{eq:BRc1sol}
 T^{\mu\nu} = \Theta^{\mu\nu}+ \partial_\lambda \psi^{\lambda\mu\nu}\,.
\end{equation}
%As our aim is to symmetrise $\Theta$, we may assume $\psi^{\lambda\mu\nu} = -\psi^{\lambda\nu\mu}$. -- lófaszt.

If eq.\ (\ref{eq:BRc1sol}) holds, the integral of $\partial\psi$ in eq.\ (\ref{eq:BRc1}) is a surface term, which vanishes if the fields decay rapidly enough. Eq.\ (\ref{eq:BRc2}) is satisfied if
\begin{equation}\label{eq:BRc2sol}
\partial_\lambda \left(\psi^{\lambda\mu\nu}-\psi^{\lambda\nu\mu}\right) = -\left(\Theta^{\mu\nu}-\Theta^{\nu\mu}\right)\,.
\end{equation}

Conservation of $T^{\mu\nu}$, eq.\ (\ref{eq:BRc3}), holds, as $\Theta^{\mu\nu}$ is conserved [see eq.\ (\ref{eq:momcons})], if
\begin{equation}\label{eq:BRc3a}
 \partial_\mu\partial_\lambda \psi^{\lambda\mu\nu} = 0\,,
\end{equation}
which may be guaranteed if
\begin{equation}\label{eq:BRc3sol}
 \psi^{\lambda\mu\nu} = -\psi^{\mu\lambda\nu}\,.
\end{equation}

Let us notice, that eq.\ (\ref{eq:BRc2sol}) is satisfied if
\begin{equation}\label{eq:BRc2sol2}
 \psi^{\lambda\mu\nu}-\psi^{\lambda\nu\mu} = S^{\lambda\mu\nu}\,,
\end{equation}
and this, together with antisymmetry in the first two indices, determines $\psi$ uniquely,
\begin{equation}\label{eq:BRsol}
 \psi^{\lambda\mu\nu} = \frac{1}{2}\left(S^{\lambda\mu\nu} -S^{\mu\lambda\nu}+S^{\nu\mu\lambda}\right)\,.
\end{equation}


It is possible to introduce an angular momentum density using the symmetrised stress-energy tensor $T^{\mu\nu}$ as
\begin{equation}\label{eq:angmomBR}
 M^{\lambda\mu\nu}_{\rm BR} = x^\mu T^{\lambda\nu} - x^\nu T^{\lambda\mu}\,.
\end{equation}
The difference between $M^{0\mu\nu}_{\rm BR}$ and $M^{0\mu\nu}$ is $\partial_\lambda(x^\mu \psi^{\lambda0\nu} - x^\nu \psi^{\lambda0\mu})$, using the atisymmetry of $\psi$ in the first two indices, this is seen to be a surface term, therefore, the angular momentum densities (\ref{eq:angmomBR}) and (\ref{eq:AM}) yield the same integrated angular momentum, provided that the fields decay rapidly enough at large distances.


\section{Applications}\label{sec:appl}
\subsection{Real scalar field}\label{ssec:scalar}
A real scalar field interacting with a source $\sigma$ is described by the Klein-Gordon Lagrangian
\begin{equation}\label{eq:LagSc}
 \lag_{\rm KG} = \frac{1}{2}\partial_\mu \phi \partial^\mu \phi -\frac{1}{2}m_s^2 \phi^2 -\sigma\phi\,.
\end{equation}
The field equation is the Klein-Gordon equation,
\begin{equation}\label{eq:KGe}
 \partial_\mu \partial^\mu \phi + m_s^2 \phi = -\sigma\,.
\end{equation}
In this case, the spin tensor vanishes, and the canonical energy-momentum tensor is symmetric,
\begin{equation}\label{eq:EMcanSc}
 T^\mu_{\rm KG}{}_\nu = \Theta^\mu_{\rm KG}{}_\nu =\partial^\mu \phi \partial_\nu \phi - \delta^\mu_\nu \lag_{\rm KG}\,,
\end{equation}
and the spin tensor vanishes, $S^{\lambda\mu\nu}_{rm KG} = 0$.


\subsection{Complex scalar field}\label{ssec:cscalar}
A complex scalar field may be considered simply two real ones with a global $U(1)=SO(2)$ symmetry, $\phi=\phi_1+\imagi\phi_2$, and therefore, its Lagrangian is also just the sum of the Lagrangians of the two componets, which may be written as
\begin{equation}\label{eq:LagCSc}
 \lag_{\rm cs}=\partial_\mu \phi^* \partial^\mu\phi -m_s^2 \phi^*\phi-\sigma\phi^*-\sigma^*\phi\,.
\end{equation}
Equations of motion are derived with the usual fashion, or, by using Wirtinger calculus, often referred to as ``treating $\phi$ and $\phi^*$ as independent variables'',  defining $\partial/\partial\phi = (\partial/\partial\phi_1 - \imagi\partial/\partial\phi_2)/2$ and $\partial/\partial\phi^* = (\partial/\partial\phi_1 + \imagi\partial/\partial\phi_2)/2$, and using these for variational calculus\footnote{Note, that the condition for complex differentiability, the Cauchy-Riemann relations for a function $f(z)$ may be written as $\partial f(x)/\partial z^*=0$, and, e.g., $|z|^2=z^*z$ is not complex differentiable, but, of course, $\mathbb{R}^2$-differentiable, $\partial (z^*z)/\partial z^*=z$, $\partial (z^*z)/\partial z=z^*$.}, yielding
\begin{equation}\label{eq:cKGe}
 \partial_\mu\partial^\mu \phi -m_s^2\phi=-\sigma\,,\quad
 \partial_\mu\partial^\mu \phi^* -m_s^2\phi^*=-\sigma^*\,.
\end{equation}
Again, the canonical and the symmetrised stress-energy tensors agree,
\begin{equation}\label{eq:EMcanCSc}
 \Theta^{\mu\nu}_{\rm cs} = T^{\mu\nu}_{\rm cs}= \partial^\mu\phi^*\partial^\nu\phi + \partial^\mu\phi\partial^\nu\phi^*-g^{\mu\nu}\lag_{cs}\,,
\end{equation}
which is also the sum of the stress-energy tensors of the components. The only novelty is the conserved current corresponding to the symmetry $\phi\to \e^{i\alpha}\phi$ ($\alpha$ real), under an infinitesimal transformation $\alpha=\epsilon$, $\delta\phi=\imagi\epsilon\phi$, $\delta\phi^*=-\imagi\epsilon\phi^*$, or, $M^._. = \imagi$ $M^*_*=-\imagi$, yielding
\begin{equation}\label{eq:JISCSc}
 j_{{\rm cs}\,\mu} = \imagi(\phi\partial_\mu\phi^* - \phi^*\partial_\mu\phi)\,.
\end{equation}

Let us also consider derivative coupling, we shall couple $j^{{\rm cs}\,\mu}$ to a vector field $A^\mu$, with the interaction Lagrangian $-ej_{{\rm cs}\,\mu}A^\mu+2e^2\phi^*\phi A_\mu A^\mu$. This form corresponds to replacing the partial derivatives in the current (\ref{eq:JISCSc}) with gauge covariant derivatives, and coupling that to $A^\mu$ as $-e A_\mu j^\mu_{{\rm cs}\,\partial \to D}$, where we have introduced the \emph{gauge covariant derivative} $D_\mu \phi = (\partial_\mu-\imagi e A_\mu)\phi$.

In this case, there is an addition to $\Theta^\mu_{{\rm sc}\,\nu}$, $\imagi e A^\mu \phi^*\partial_\nu \phi + c.c.$, yielding an asymmetric canonical energy-momentum tensor,
\begin{equation}\label{eq:EMcanCScE}
 \Theta^\mu_{{\rm cs,i}\,\nu} = D^\mu\phi^*\partial_\nu\phi + D^\mu\phi\partial_\nu\phi^* -\delta^\mu_\nu \lag_{\rm cs,i}\,,
\end{equation}
where $\lag_{\rm cs,i} = \lag_{\rm cs}-e A_\mu j^\mu_{\rm cs,i}$, and $j^\mu_{\rm cs,i}=j^\mu_{\partial \to D}$.

Taking into account the modification to $\partial_\lambda \psi^{\lambda\mu\nu}$ in the case of the electromagnetic stress-energy tensor, we obtain
\begin{equation}\label{eq:SymmCSci}
 T^{\mu\nu}_{\rm cs,i} = D^\mu\phi^* D^\nu\phi + D^\mu\phi D^\nu\phi^* - g^{\mu\nu}\lag_{\rm cs,i}\,,
\end{equation}
and for the full system, we need to add $T^{\mu\nu}_{\rm em}$ without the modifications due to the interaction. The same result holds for interaction with massive electrodynamics (Proca field).





\subsection{Electrodynamics}\label{ssec:eldyn}
The Lagrangian of the Maxwell field is
\begin{equation}\label{eq:LagM}
 \lag_{\rm em} = -\frac{1}{4}F_{\mu\nu}F^{\mu\nu} -j_\mu A^\mu\,,\quad F_{\mu\nu}=\partial_\mu A_\nu - \partial_\nu A_\mu\,,
\end{equation}
where $j^\mu$ is a source vector field, which we shall assume to be conserved, $\partial_\mu j^\mu =0$.

Field equations are Maxwell's equations,
\begin{equation}\label{eq:Maxwell4}
 \partial_\mu F^{\mu\nu} = j^\nu\,,\quad\text{or}\quad \partial_\nu F^{\mu\nu}=-j^\mu\,.
\end{equation}
The canonical stress-energy tensor is then for pure (source free) Maxwell field ($j^\mu=0$)
\begin{equation}\label{eq:EMcanM}
 \Theta^\mu_{\rm em}{}_\nu = -F^{\mu\lambda}\partial_\nu A_\lambda +\frac{1}{4}\delta^\mu_\nu F_{\lambda\rho}F^{\lambda\rho}\,.
\end{equation}
For the spin tensor one obtains
\begin{equation}\label{eq:SpinAMM}
S^{\lambda\mu\nu}_{\rm em}  = A^\mu F^{\lambda\nu}-A^\nu F^{\lambda\mu}\,.
\end{equation}
The Belinfante-Rosenfeld tensor is thus from eq.\ (\ref{eq:BRsol})
\begin{equation}\label{eq:BRsolM}
 \psi_{\rm em}^{\lambda\mu\nu} = A^\nu F^{\mu\lambda}\,,
\end{equation}
which, when added to the canonical tensor in eq.\ (\ref{eq:EMcanM}) yields
\begin{equation}\label{eq:SymmM}
 T^{\mu\nu} = -F^{\mu\lambda}F_{\nu\lambda} +\frac{1}{4}g^{\mu\nu} F_{\lambda\rho}F^{\lambda\rho}\,,
\end{equation}
as $\partial_\lambda \psi^{\lambda\mu\nu} = \partial_\lambda A^\nu F^{\mu\lambda}$, as $\partial_\lambda F^{\mu\lambda}=0$ for source-free electromagnetic fields. (Note the typo here in Ref.\ \cite{LL2}.)

In the case when sources are included, $\delta^\mu_\nu j_\lambda A^\lambda$ is added to $\Theta^{\mu}_{\rm em}{}_\nu$ and $-A^\nu j^\mu +g^{\mu\nu} j_\lambda A^\lambda$ to $T^{\mu\nu}_{\rm em}$. (Note, that our calculation was based on the assumption that the Lagrangian is translation-independent, i.e., we are assuming here, that the sources also satisfy some field equations originating from a translation-invariant Lagrangian.)

\subsection{Proca field}\label{ssec:Proca}
Proca theory is massive electrodynamics, with the Lagrangian
\begin{equation}\label{eq:LagP}
 \lag_{\rm P} = -\frac{1}{4}F_{\mu\nu}F^{\mu\nu} +\frac{1}{2}m_A^2 A_\mu A^\mu -j_\mu A^\mu\,,\quad F_{\mu\nu}=\partial_\mu A_\nu - \partial_\nu A_\mu\,,
\end{equation}
where $j^\mu$ is a source vector field, which we shall assume to be conserved, $\partial_\mu j^\mu =0$.

The field equations are Maxwell's equations modified by the mass term,
\begin{equation}\label{eq:Proca4}
 \partial_\mu F^{\mu\nu} + m_A^2 A^\nu= j^\nu\,,\quad\text{or}\quad \partial_\nu F^{\mu\nu}-m_A^2 A^\mu=-j^\mu\,.
\end{equation}

The canonical stress-energy tensor is then
\begin{equation}\label{eq:EMcanP}
 \Theta^\mu_{\rm P}{}_\nu = -F^{\mu\lambda}\partial_\nu A_\lambda +\frac{1}{4}\delta^\mu_\nu F_{\lambda\rho}F^{\lambda\rho} - \frac{1}{2}\delta^\mu_\nu m_A^2 A_\lambda A^\lambda\,.
\end{equation}
The spin tensor is the same as in electrodynamics
\begin{equation}\label{eq:SpinAMP}
S^{\lambda\mu\nu}_{\rm P}  = A^\mu F^{\lambda\nu}-A^\nu F^{\lambda\mu}\,,
\end{equation}
and, therefore, so is the Belinfante-Rosenfeld tensor,
\begin{equation}\label{eq:BRsolP}
 \psi_{\rm em}^{\lambda\mu\nu} = A^\nu F^{\mu\lambda}\,,
\end{equation}
which, when added to the canonical tensor in eq.\ (\ref{eq:EMcanP}) yields
\begin{equation}\label{eq:SymmP}
 T^{\mu\nu} = -F^{\mu\lambda}F_{\nu\lambda}+m_A^2 A^\mu A^\nu +\frac{1}{4}g^{\mu\nu} F_{\lambda\rho}F^{\lambda\rho}- \frac{1}{2}g^{\mu\nu} m_A^2 A_\lambda A^\lambda\,,
\end{equation}
as $\partial_\lambda \psi^{\lambda\mu\nu} = \partial_\lambda A^\nu F^{\mu\lambda}+m_A^2 A^\nu A^\mu$, as $\partial_\lambda F^{\mu\lambda}=m_A^2 A^\lambda$ for source-free electromagnetic fields.

An interesting feature of the energy density of the Proca field is that the coefficient of $A_0^2$ in the canonical energy density $\Theta^{00}$ is negative, $-m_A^2/2$. In the Belinfante-Rosenfeld energy density, $T^{00}$ it is positive $m_A^2/2$. Thus, definiteness of energy (the integral) is proven, ad the integral of the canonical and the symmetrised energy-momentum densities agree. This also shows, that Belinfante-Rosenfeld symmetrisation is not simply algebraic symmetrisation.


In the case when sources are included, $\delta^\mu_\nu j_\lambda A^\lambda$ is added to $\Theta^{\mu}_{\rm P}{}_\nu$ and $-A^\nu j^\mu +g^{\mu\nu} j_\lambda A^\lambda$ to $T^{\mu\nu}_{\rm P}$. The remarks when considering interaction terms in Sec.\ \ref{ssec:eldyn} apply here as well.


\subsection{Gauge kinetic mixing}\label{ssec:gkm}
Let us consider now a theory with two gauge fields, $A_1^\mu$, $A_2^\mu$. A possible interaction is the gauge kinetic mixing, proposed in Ref.\ \cite{holdom},
\begin{equation}\label{eq:laggkm}
 \lag_{\rm GKM} = \frac{\sin\epsilon}{2}F_{1,\mu\nu}F_2^{\mu\nu}\,,
\end{equation}
where $F_{i}$ are the field strenght tensors corresponding to $A_i$ with the usual formulae. We shall assume that the gauge fields both appear in the Lagrangian as $A^\mu$ in Eq.\ (\ref{eq:LagM}).

The canonical stress-energy tensor, in addition to the separate contributions of the two fields receives a contribution
\begin{equation}\label{eq:EMcanGKM}
\Theta^\mu_{{\rm GKM}\,\nu}  = \sin\epsilon (F^{\mu\lambda}_1 \partial_\nu A_{2,\lambda} + F^{\mu\lambda}_2 \partial_\nu A_{1,\lambda}) -\delta^\mu_\nu \lag_{\rm GKM}\,.
\end{equation}
Similarly, the spin tensor receives the correction
\begin{equation}\label{eq:SpinGKM}
 S^{\lambda\mu\nu}_{\rm GKM} = -\sin\epsilon(A^\mu_1 F^{\lambda\nu}_2 -A^\nu_1 F^{\lambda\mu}_2 + (1\leftrightarrow 2))\,.
\end{equation}
and
\begin{equation}\label{eq:BRGKM}
\psi_{\rm GKM}^{\lambda\mu\nu} = -\sin\epsilon(A^\nu_1 F_2^{\mu\lambda}+A^\nu_2 F_1^{\mu\lambda})\,, 
\end{equation}
and, finally,
\begin{equation}\label{eq:SymmGKM}
 T^{\mu\nu}_{\rm GKM} = \sin\epsilon ( F^{\mu\lambda}_1 F^\nu_{2,\lambda} +F^{\mu\lambda}_2 F^\nu_{1,\lambda})-g^{\mu\nu}\lag_{\rm GKM}\,.
\end{equation}









\subsection{The Dirac field}\label{ssec:Dirac}
The Dirac field satisfies field quations derived from the Lagrangian
\begin{equation}\label{eq:LagD}
 \lag_{\rm D} =\bar{\psi}(\imagi\overleftrightarrow{\slashed{\partial}}-m)\psi\,,
\end{equation}
where $\slashed{\partial}=\partial_\mu\gamma^\mu$, $\gamma^\mu$ are the Dirac matrices (in some representation), $\bar\psi=\psi^\dagger\gamma^0$, and $f\overleftrightarrow{\partial_\mu}g = (1/2)(f\,\partial_\mu g-\partial_\mu f\,g)$. In what follows, we shall pay attention not to change ordering of Dirac fields, so that the results remain valid for ordinary spinors and anticommuting fields.

The field equations corresponding to the Lagrangian (\ref{eq:LagD}) are, by varying independently w.r.t.\ $\psi$ and $\psi^\dagger$, using $(\gamma^0)^2=1$,
\begin{equation}\label{eq:Dirac}
 (\imagi\slashed{\partial}-m)\psi=0\,,\quad\quad \bar{\psi}(-\imagi\overleftarrow{\slashed{\partial}}-m)=0\,,
\end{equation}
where $f\overleftarrow\partial=\partial f$.

The canonical stresss-energy tensor is
\begin{equation}\label{eq:EMcanD}
\Theta^\mu_{{\rm D}\nu}= \imagi \bar{\psi} \gamma^\mu \overleftrightarrow{\partial}_\nu\psi -\delta^\mu_\nu \lag_{\rm D}\,.
 \end{equation}
Note, that on shell, i.e., using the field equation (\ref{eq:Dirac}), the Lagrangian vanishes.
 
The spin tensor is
\begin{equation}\label{eq:SpinD}
 S^{\lambda\mu\nu}_{\rm D} = \frac{1}{4}\bar{\psi}\left\{\sigma^{\mu\nu},\gamma^\lambda\right\}\psi = 2\varepsilon^{\sigma\mu\nu\lambda}\bar{\psi}\gamma_\sigma \gamma^5\psi\,,
\end{equation}
where $\gamma^5=\imagi \gamma^0\gamma^1\gamma^2\gamma^3$, and $\epsilon^{\mu\nu\lambda\rho}$ is the fully antisymmetric unit tensor, normalised as $\epsilon^{0123}=1$.

The spin tensor (\ref{eq:SpinD}) yields the Belinfante-Rosenfeld tensor
\begin{equation}\label{eq:BRsolD}
 \psi^{\lambda\mu\nu}_{\rm D} = \frac{1}{2}S^{\lambda\mu\nu}_{\rm D}
\end{equation}
for the Dirac field, which can be shown, e.g., by using the properties of the $\varepsilon$ tensor.

The calculate the divergence of the Belinfante-Rosenfeld tensor, we write out the anti-commutator, and use the commutator
\[
\left[ \sigma^{\mu\nu},\gamma^\lambda \right] = 2\imagi\left( g^{\lambda\nu}\gamma^\mu - g^{\lambda\mu}\gamma^{\nu}\right)
\]
in all terms where the $\partial_\lambda$ derivative is not next to $\gamma^\lambda$, and the Dirac and adjoint Dirac equations (\ref{eq:Dirac}), to cancel thos terms, remaining
\[
 \partial_\lambda \psi^{\lambda\mu\nu}_{\rm D} = \frac{\imagi}{4}(-\partial^\mu \bar{\psi} \gamma^\nu \psi + \partial^\nu \bar{\psi}\gamma^\mu\psi + \bar{\psi}\gamma^\nu \partial^\mu \psi -  \bar{\psi}\gamma^\mu \partial^\nu \psi)\,,
\]
adding which to the canonical stress-energy tensor (\ref{eq:EMcanD}) yields
\begin{equation}\label{eq:SymmD}
 T^{\mu\nu}_{\rm D} = \frac{\imagi}{2}\left( \bar{\psi} \gamma^\mu \overleftrightarrow{\partial}{}^\nu \psi + \bar{\psi} \gamma^\nu \overleftrightarrow{\partial}{}^\mu \psi\right)\,.
\end{equation}

It is also possible to couple the Dirac field to the electromagnetic field via minimal coupling, replacing $\partial_\mu$ by $D_\mu = \partial_\mu -\imagi e A_\mu$ in the Dirac Lagangian (\ref{eq:LagD}). Firstly, this couples the electromagnetic field to the Dirac current,
\begin{equation}
j^\mu_{\rm D} = -e \bar{\psi}\gamma^\mu \psi\,,
\end{equation}
which is also the Noether current corresponding to the $U(1)$ symmetry of the Lagrangian (\ref{eq:LagD}) w.r.t.\ the transformation $\psi \to \exp(\imagi e \alpha)\psi$, $\bar\psi \to \exp(-\imagi e \alpha)\bar\psi$.
The interaction Lagrangian is then $\lag_{\rm int} = -j^\mu_{\rm D}A_\mu$.

Due to the fact, that no derivative of the Dirac field is coupled to the electromagnetic potential, the canonical stress-energy tensor (\ref{eq:EMcanD}) is unchanged upon adding the coupling. From the $\partial_\lambda \psi^{\lambda\mu\nu}$ term, a new contribution arises,
\[
 T^{\mu\nu}_{\rm D,int} = \frac{1}{2}A^\nu j^\mu_{\rm D}-\frac{1}{2}A^{\mu}j^\nu_{\rm D}\,,
\]
which, together with the correction to the same term in the electromagnetic case, $-A^\nu j^\mu$, yieds a symmetric contribution, $-(1/2)(A^\mu j^\nu_{\rm D} + j^\mu_{\rm D}A^\nu)$. The sum agrees with $T^{\mu\nu}_{\rm em}+T^{\mu\nu}_{{\rm D},\partial\to D}$, i.e., replacing derivatives with covariant derivatives in the symmetric Dirac stress-energy tensor (and using $\overleftrightarrow{\slashed{D}} = (\overleftrightarrow{\partial}_\mu -\imagi e A_\mu)\gamma^\mu$).


%Quite remarkably, the stress-energy tensor of the interacting Maxwell-Dirac system is



%using $\partial_\lambda \psi^{\lambda\mu\nu}=(\imagi/2)\partial^\nu \left(\bar{\psi}\gamma^\mu\psi\right)$, deriving which we have used the Dirac equation and its adjoint, eqs.\ (\ref{eq:Dirac}).





\begin{thebibliography}{99}
\bibitem{LL2} L.D.~Landau and E.M.~Lifshitz, \emph{Course of theoretical physics 2: The classical theory of fields} (Pergamon Press, Oxford, UK, 1975).

\bibitem{wentzel} G.~Wentzel, \emph{Quantum theory of fields} (Interscience Publishers, New York, USA, 1949).

\bibitem{BD} J.D.~Bjorken and S.D.~Drell, \emph{Relativistic quantum mechanics} (McGraw-Hill Book Co., New York, NY, USA, 1964).

\bibitem{itzykson} C.~Itzykson and J.-B.~Zuber, \emph{Quantum field theory} (Dover Publications, Mineola, NY, USA, 2005).

\bibitem{holdom} B.~Holdom, \emph{Two $U(1)$'s and $\epsilon$ charge shifts}, {\sl Phys.\ Lett.} {\bf 166B}, 196-199 (1986).

\end{thebibliography}

\end{document}
